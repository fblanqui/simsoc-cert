\section{Conclusion}
\label{sec:conclusion}

The technique introduced in \cite{small_inv}
on very small toy examples
could be successfully used in a significant application,
up to suitable extensions in order to conveniently get
the premises of a constructor in non-absurd cases.
As in \cite{small_inv},
we don't claim that we have a fully automated tactic,
like \inversion.
Our goal is more modest:
providing a hand-crafted inversion technique 
which is flexible enough for the user,
so that most practical situations can be managed
with a full control on the script and valuable
improvements on robustness.
Moreover, the extra flexibility provided by hand-crafted inversions can
be exploited to produce smaller, more manageable proof terms.

Our method was experimented on large proofs relying on 
big inductive relations independently defined in the Compcert project.

The current development can be found on-line\footnote{%
\label{f:site}
%\url{http://formes.asia/media/simsoc-cert/}}
\url{http://www-verimag.imag.fr/~monin/Proof/SmallInvScalesUp/}
}.

Our group recently started another project dedicated
to a certifying compiler from a high-level component-based language dedicated
to embedded systems (BIP), with CompCert C as its target. 
We expect the work presented here and our high-level tactics
to be reused there.

Let us mention another possible application of the technique.
Inversion is sometimes
needed to write a function whose properties will be established later (as
opposed to providing a monolithic and exhaustive Hoare-style specification and
along with a VCC generator such as Program). 
In this context simply using the proof engine and the \inversion tactic
tends to generate unmanageably large terms.
We can expect our technique could be very helpful in such situations.


% and the line-count metrics given at the end of \cite{small_inv} makes sense



%%% Local Variables: 
%%% mode: latex
%%% TeX-master: "cpp12"
%%% End: 
