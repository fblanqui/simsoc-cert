\section{Handling positive cases}
\label{sec:improvement}

However, 
the technique explained in the previous section 
has to be extended in order to cover more general
situations. 

A first easy improvement makes \diag independent
from the conclusion.
To this effect, we replace it with $(\forall X, X)$ 
in the first branch of \diag.
In our previous example, this yields

\coqdocinput{chunk27}

\noindent
Then the previous proof term 
(\match $H$ \intac $\eveni\;n$ \retac $\diag\;n$ \with 1 $\ldots$)
has the type $\forall X, X$
and then can be successfully applied to any current conclusion.
Alternatively, we can define a general function as follows:

\coqdocinput{chunk29}


\medskip
Next consider the following theorem:

\coqdocinput{chunk28}

\noindent
The proof is by induction on $\eveni\;n$.
In the inductive step, we have to prove $\eveni\;m$
from the induction hypothesis $\eveni\:(n + m) \rightarrow \eveni\;m$
and a new hypothesis $H: \eveni\: (S\: (S\: (n + m)))$.
Intuitively, we want to invert $H$ in order to push $\eveni\:(n + m)$
in the environment. 
The trick given at the end of section~\ref{sec:absurd}
is then of no help.
However, we can instead adapt \prone as follows:

\coqdocinput{chunk30}

\noindent
Then, applying \prET to $H$ yields a function in continuation passing style.
Its type parameter $X$ is automatically identified to the conclusion
$\eveni\;m$, while $y$ is bound to $n+m$,
so that we get a new goal $\eveni\:(n+m) \rightarrow \eveni\;m$.
That is, we have exactly the expected inversion.
Functions such as \prone and \prET can be seen as inversion
lemmas, but note that their type is the dependent type
expressed by their own \diag.

More generally,
in the case of an inductive type $A(u)$ with parameter $u:U$,
given a constructor of type $\forall \mathbf{p}, A \;p$, 
where $\mathbf{p}$ is a telescope and $p:U$ an expression made
of constructors in the type $U$,
we proceed similarly:
the \match of \diag has a first branch filtering $\cal P$
and returning 
$\forall X: Prop, (\forall \mathbf{p}, X) \rightarrow X$.
If several constructors are possible for $A \:\cal P$,
say $C_1: \forall \mathbf{p_1}, A \:\cal P$,
$\ldots$, $C_n: \forall \mathbf{p_n}, A \:\cal P$,
the inverting lemma corresponding to $A \:\cal P$ will be:

\coqdocinput{chunk19}

\noindent
Remark the close relationship with the impredicative encoding
of data-types in system F.


%%%% END of JF, BEGIN of XM

The next stage to be considered is the case of
an inductive type with more than one parameter.
This raises new issues, because additional identities
between arguments of the premises or the conclusion
of a constructor may occur.
This happens routinely in our main experiment,
to be described below in section~\ref{sec:simsoccert}.
In order to explain the problems and how to deal with
them in our framework, 
we introduce a toy language.

% To change to this new example is closer to what we have during using CompCert C
% language semantics in SimSoC-Cert project.
% It helps to explain the next problems we met.

The following code is the Coq description of this toy language.
The inductively defined evaluation rule $eval$ has two parameters.
The first one is the input type $tm$, $tm\_const$ and $tm\_plus$ are the
expected cases in pattern matching.
The second parameter is an output of type $val$ is either nat or bool.
%and it is the extra variable we have to deal with.

\medskip
\coqdocinput{chunk31}
\medskip

In constructor $E\_Plus$,
the two premises share the variables $t1, t2, n1, n2$ with the
conclusion. 
If we use the last solution 
with continuation passing style, as it is presented above,
we are able to keep the premises 
but the relationship between the output values
as specified in the inductive definition will be lost
in the generated subgoal.

In order to solve this issue,
we add a suitable parameter to $X$ 
corresponding to the second parameter of the inductive relation.
The function for extracting the premises of $E\_Plus$ is:

\medskip
\coqdocinput{chunk34}
\medskip

Consider the following example.

\medskip
\coqdocinput{chunk38}
\medskip

By applying $pr\_plus$, $v$ will be equated to $nval~(plus~n1~n2)$ 
according to the rule specified by $E\_plus$.

% Next, consider an extra hypothesis $P\; v$ taking the same variable $v$. 
% This $v$ needs to be assigned to $nval~(plus~n1~n2)$ too.
% So except the inductive predicate we want to invert, 
% all other hypothesis share the same parameters which need to be assigned by
% equivalence relation, should be first revert to the goal.

% The second thing is when other hypothesis contain the same variables.
% Hypothesis $eval (tm\_plus (tm\_const 1) (tm\_const 0)) v$ is
% our inverting target, where $v$ is the output value we want to keep the.
% And hypothesis $P v$ is taking varialbe $v$ as parameter
% We want all 

% In order to have them changed to the same value as in the goal when perform
% applying rules. So we need Coq tactic \coqdockw{revert} to put it in goal before
% apply a corresponding definition.

% The third problem to solve is,
% according to what we have in proof goal we may need to provide different
% diagonalization function for some constructor.


% Then we want to have the equility $n1 = 1$ and $n2 = 0$.
% Normally, we will define a diagonalization function having the similar form 
% as $pr\_plus$ using a $X$ of type $val -> Prop$.

% \medskip
% \coqdocinput{chunk32}
% \medskip

% But the fact is
% applying such function will keep the equivalence between $nval n$ and
% $nval 1$. Then the proof assistant will complain that in goal we have nothing
% in form of $nval n1$ but in form of $nval (n1 + n2)$ with operator $+$.
% Then we know in this case,
% the variable we want to keep is not $nval n1$ but $n1$ which is of type 
% $nat$. But you can't have this $n1$ for free.
% So the diagonalization function will match two parameters: the key parameter
% matching type $tm$ of expected result $tm\_const tc$,
% and the evaluation output $v$ of $nval n$ in order to
% register $n$ independently.

% \medskip
% \coqdocinput{chunk33}
% \medskip

% And the other diagonalization function $pr\_const\_1$ will be used
% when the goal is in normal form $nval n$,
% in this case $pr\_const\_1\_2$ won't be able to help.




%%% Local Variables: 
%%% mode: latex
%%% TeX-master: "cpp12"
%%% End: 
