\selectlanguage{french}
\chapter*{Version française\markboth{Version française}{}}
\addcontentsline{toc}{chapter}{Version française}
\renewcommand{\thechapter}{F}
\setcounter{table}{0} 

\section*{Introduction}
\addcontentsline{toc}{section}{Introduction}

\subsection*{Certification de SimSoC}
\addcontentsline{toc}{subsection}{Certification de SimSoC}

%
Cette thèse expose nos travaux de certification d'une partie d'un programme C/C++
nommé \simsoc (Simulation of System on Chip), %~\cite{helmstetter2008simsoc},
qui simule le comportement d'architectures basées sur des processeurs
tels que ARM, PowerPC, MIPS ou SH4.

Un simulateur de \emph{System on Chip} peut être utilisé pour développer
le logiciel d'un système embarqué spécifique, afin de raccourcir les phases
des développement et de test, en particulier quand la vitesse de simulation
est réaliste (environ 100 millions d'instructions par seconde par c{\oe}ur
dans le cas de \simsoc).
Les réductions de temps et de coût de développement obtenues se traduisent
par des cycles de conception interactifs et rapides,
en évitant la lourdeur d'un système de développement matériel.

Un problème critique se pose alors : le simulateur simule-t-il effectivement
le matériel réel ?
Pour apporter des éléments de réponse positifs à cette question,
notre travail vise à prouver la correction d'une partie significative
de \simsoc, de sorte à augmenter la confiance de l'utilisateur en ce simulateur
notamment pour des systèmes critiques.

\simsoc est un logiciel complexe,
comprenant environ 60\:000 de C++,
intégrant des parties écrites en SystemC et
des optimisations non triviales pour atteindre une grande vitesse de simulation.
%
La partie de \simsoc dédiée au processeur ARM,
l'un des plus répandus dans le domaine des SoC,
transcrit les informations contenues
dans un manuel épais de 1138 pages.
La première version du simulateur ARM de \simsoc a été codée à la main.
Les erreurs sont inévitables à ce niveau de complexité,
et certaines sont passées au travers des tests intensifs
effectués sur la version précédente de \simsoc pour l'ARMv5,
qui réussissait tout de même à simuler l'amorçage complet de Linux.
Au delà des objectifs de vitesse,
la précision est nécessaire.
Toutes les instructions doivent être simulées exactement comme décrit
dans la spécification (en supposant que le matériel en fait de même).
Dans les expériences effectuées avec \simsoc, des erreurs de comportement
sont apparues de temps à autre mais il était très difficile 
d'en détecter l'origine. 
Des tests intensifs peuvent couvrir la plupart des instructions
mais certains cas rares restent inexplorés, laissant la voie ouverte
à des problèmes potentiels.

Une meilleure approche est alors nécessaire pour atteindre 
un degré de confiance plus élevé dans le simulateur. 
Nous proposons de certifier \simsoc par l'emploi de méthodes formelles.

Dans cette thèse, nous considérons l'un des modules de \simsoc:
le simulateur d'architecture ARM.
L'architecture ARM est l'une des plus répandues sur le marché
des systèmes embarqués, en particulier les téléphones mobiles et les tablettes.
Selon la compagnie ARM, 6,1 milliards processeurs ARM ont été mis sur le marché en 2010, et 95\:\% d'entre eux sont utilisés dans des \emph{smart phones}.

Comme indiqué plus haut, le simulateur représente un gros volume de logiciel.
Et sa spécification elle-même est assez complexe, 
du fait de l'architecture sophistiquée de l'ARM, 
qui comporte de nombreux composants.
Avant de considérer l'ensemble de l'architecture ARM,
nous avons décidé de concentrer nos efforts 
sur les parties essentielles,
qui sont à la fois les plus importantes et les plus sensibles:
celles qui concernent le CPU (dans la famille de processeurs ARM11).
Au moment où notre travail a démarré, le module de simulation de l'ARM
réalisé dans \simsoc correspondait à la version 5 (ARMv5).
Plutôt que de certifier cette architecture bientôt dépassée,
nous avons décidé d'anticiper l'évolution de \simsoc
et de travailler sur la version suivante: ARM Version 6 (ARMv6).
Pour des raisons expliquées plus bas,
liées aux technologies de vérification disponibles pour C
(en particulier \compcert),
il était nécessaire de travailler sur un module écrit en C plutôt qu'en C++.
Ce module, appelé \simlight, peut s'exécuter soit seul,
soit en tant que composant intégré dans \simsoc.

Plus de 60\:\% du développement de \simsoc est constitué de la partie CPU
(voir figure~\ref{fig:dev} page~\pageref{fig:dev}).
Les parties restantes comprennent la gestion mémoire (mémoire virtuelle
et pagination), la gestion des interruptions et les communications avec
les périphériques.
En résumé, la complexité de la cible a ainsi pu être réduite
mais cela représente encore 10\,000 lignes de code C à certifier.
En outre la spécification reste très complexe,
étant décrite dans un gros document, 
le manuel de référence de l'architecture ARM Version~6.

L'enjeu est donc ici de certifier un programme de cette taille
par rapport à une spécification assez complexe.

Rappelons tout d'abord que la certification de programme
s'appuie toujours sur un modèle formel du programme étudié.
Un tel modèle formel est lui-même dérivé d'une sémantique formelle
du langage de programmation utilisé.
Pour des langages de programmation impératifs comme C,
on considère souvent des outils basés sur une sémantique axiomatique
(à la Hoare) comme \framac~\cite{canet2009value},
qui intègre un ensemble d'outils d'analyse de programmes pour C.
La plupart de ces outils reposent sur ACSL (ANSI/ISO C Specification Language),
un langage de spécification basé sur une sémantique axiomatique de C.
ACSL est assez puissant pour exprimer des propriétés directement
au niveau du programme C.
Des étiquettes d'état peuvent être insérées pour dénoter un point de
de contrôle du programme, et peuvent être utilisées
dans des fonctions logiques ou des prédicats.
\framac est déjà une plate-forme assez mûre pour effectuer
des analyses statiques et de la vérification déductive automatique
sur des programmes C.
Un avantage de \framac et des outils similaires est qu'il est supporté
par des technologies de preuve automatique,
qui permettent une importante économie d'efforts de la part de l'utilisateur.
Des succès ont été obtenues sur des programmes complexes et astucieux,
par exemple l'algorithme de Schorr-Waite qui manipule des structures chaînées.

% Les approches basées sur une sémantique axiomatique
% (logique de Hoare par exemple)
% sont les plus répandues en preuve de programmes impératifs.
Cependant, d'une manière générale, un haut degré d'automatisation
tend à affaiblir le niveau de la certification du fait 
que les prouveurs automatiques sont eux-mêmes des programmes
complexes et sujets à erreurs.
En théorie, de tels programmes peuvent produire des certificats
pouvant être contrôlés par un assistant à la preuve fiable
(par exemple basé sur une architecture LCF).
Mais actuellement c'est encore loin d'être le cas en pratique.
Un problème supplémentaire se trouve dans la distance qui sépare
la sémantique axiomatique de l'implantation effective,
à moins que le générateur de VC soit lui-même certifié.
Cette question a été examinée récemment, notamment dans le
travail de Paolo Herms sur \whyCert~\ref{sec:whycert}
-- et qui n'était pas disponible au moment où nous avons
commencé de travailler sur la certification de \simsoc.

Un autre problème important est que l'automatisation n'est possible que sur
des théories ou logiques offrant un pouvoir d'expression limité.
Cela peut compliquer l'expression des spécifications et des propriétés
attendues au bon niveau d'abstraction,
notamment dans un contexte où la spécification est très complexe.
Actuellement, \framac implémente un sur-ensemble de la logique du premier ordre.
Une autre limitation importante pour nous est que ACSL ne permet
pas de décrire les changements de type de pointeurs (\emph{cast}).
En revanche, la sémantique opérationnelle définie pour \compcert C
(voir ci-dessous) est capable de gérer les \emph{casts} sur des pointeurs.

% HERE : The \why software
Le logiciel \why est l'un des composants les plus importants de \framac.
Il effectue un calcul de plus faible pré-condition dans le calcul de Dijkstra.
\why constitue la base de \jessie, un plugin de \framac qui compile du
code C annoté en ACSL vers le langage intermédiaire de \jessie.
Le résultat est donné en entrée au générateur 
de conditions de vérification (VC) de \why,
qui à son tour produit des formules destinées à des prouveurs
automatiques ou interactifs comme Coq.

La version 3 de \why correspond à à changement de conception en profondeur
de \why.
Il n'y a pas encore de frontal pour le langage C.
Il s'agit d'une bibliothèque standard de théories logiques
(arithmétique entière et réelle, opérations booléennes, ensembles, etc.)
ou portant sur des structures de données de programmation (tableaux,
files, tables à adressage dispersé, etc.). 
La transmission de code C annoté en ACSL vers le générateur de VC
de \why 3 s'effectue par \jessie via un code intermédiaire
en \whyML, un langage de spécification pour la spécification
et la programmation impérative.
Dans la nouvelle architecture, le langage de spécification est enrichi
de façon à supporter davantage de prouveurs automatiques.
En outre, une interface formelle est définie pour faciliter
l'accès à de nouveaux prouveurs externes. 
Ainsi, choisir \why ou \why 3 dans notre cas impliquerait
une dépendance vis-à-vis de la chaîne de transformation constituée de
\jessie et \why, pour aller de code C annoté en ACSL jusqu'aux 
conditions de vérification pour Coq.

Dans le cas de \simsoc, nous avons besoin de prendre en compte
simultanément une très grosse spécification et des astuces disponibles en C,
comme le forçage de type, qui sont utilisées dans des fonctions
délicates liées à la gestion de la mémoire.
En d'autres termes, nous avons besoin d'un cadre de travail
qui d'une part soit assez riche en mécanismes d'abstraction
pour nous permettre de gérer une telle spécification,
et d'autre part comporte une définition précise de suffisamment
de mécanismes du langage C.
Pour les raisons exposées ci-dessus, il n'était pas clair que \framac
puisse satisfaire ces exigences, même avec Coq en sortie.
Le calcul automatique de plus faibles pré-conditions ou de domaines
de variation est peu approprié à notre cas.
Nous avons besoin de vérifier des propriétés plus spécifiques
relatives à la version formelle de l'architecture ARMv6.
Cette spécification est assez complexe, par exemple en ce qui concerne
le principal type de données défini pour exprimer l'état du processeur.

Nous avons donc préféré essayer une approche
moins classique mais plus directe,
basée sur la sémantique opérationnelle de C:
cela était heureusement rendu possible en théorie depuis la formalisation
en Coq d'une telle sémantique au sein du projet \compcert.
% et mettait à notre disposition toute la puissance de Coq pour
% gérer la complexitité de la spécification.
À notre connaissance,
%au delà de la certification d'un simulateur,
il s'agit de la première expérience de preuve de correction
de programmes C à cette échelle basée sur la sémantique opérationnelle.

\subsection*{SimSoC}
\addcontentsline{toc}{subsection}{SimSoC}

Dans cette section, nous introduisons notre cible de certification:
\simsoc, un simulateur de \emph{System-on-Chip} (SoC) capable de simuler 
divers processeurs à une vitesse réaliste.
En tant que  simulateur de \emph{System-on-Chip}, les objets simulés
sont des processeurs de systèmes embarqués utilisés dans des équipements
modernes d'électronique grand public ou de systèmes industriels
(par exemple ARM, PowerPC, MIPS).
Il entre dans la catégorie des \emph{simulateurs de système complets}
car il peut simuler la plate-forme matérielle complète et exécuter
le logiciel embarqué ``tel quel'', y compris le système d'exploitation.
Ce genre de simulateur joue un rôle important dans le développement
des systèmes embarqués,
car le logiciel embarqué peut être testé et développé sur le simulateur.
Si l'on veut que le logiciel et le matériel soient prêts à aller sur
le marché en même temps, le logiciel doit parfois
être développé avant que le matériel soit disponible.
Un modèle exécutable du SoC est alors nécessaire.
Un simulateur procure d'autres avantages, permettant de combiner
la simulation avec des méthodes formelles comme le model-checking
ou l'analyse de traces pour découvrir des anomalies matérielles ou logicielles.

Notre simulateur, \simsoc, travaille au bas niveau du système.
Il prend du code binaire réel en entrée ainsi qu'un modèle de simulation
de la carte complète : processeur, unités mémoire, bus, contrôleur réseau, etc.
Il peut émuler le comportement de l'exécution des instructions,
des exceptions et des interruptions des périphériques.

En dehors du développement de logiciel, le simulateur peut aussi être utilisé
pour la conception de matériel.
En présence de composants supplémentaires fournis par une tierce partie,
les développeurs peuvent tester modulairement ces derniers dans 
l'environnement complet de simulation.

\simsoc est développé en SystemC, une bibliothèque C++,
et utilise TLM (\emph{transaction level modelling})
pour modéliser les communications entre les modèles de simulation.
Afin de simuler les processeurs à une vitesse raisonnable, 
la simulation du jeu d'instructions utilise une technique appelée
traduction dynamique (\emph{dynamic translation}),
qui traduit l'entrée binaire en une représentation intermédiaire
elle-même compilée en code de la machine hôte.
\simsoc étant un environnement assez gros et complexe qui influence
de développement à la fois de logiciel et de matériel,
il est important de comprendre quelles sont les parties les plus
significatives afin de pouvoir déterminer la cible de la certification.

\subsubsection*{Simulation du jeu d'instructions}

Un simulateur de systèmes complets doit inclure
un simulateur de jeu d'instructions, 
qui lit les instructions du programme et émule le comportement
du processeur cible.
Pour illustrer notre cible de certification,
nous détaillons ici les techniques de réalisation 
d'un simulateur de jeu d'instructions.
Il y a trois sortes de techniques implantées pour la simulation d'instructions
dans \simsoc,
correspondant à différents compromis entre précision et efficacité.
Ce sont:
la \emph{simulation par interprétation},
la \emph{traduction dynamique sans spécialisation}
et la \emph{traduction dynamique avec spécialisation}.
%
La \emph{simulation par interprétation} est la méthode classique,
qui comprend trois étapes:
récupération de l'instruction en mémoire, décodage et exécution.
Bien que cette technique soit handicapée par
le décodage répété des mêmes instructions,
elle est simple à réaliser et fiable.
Elle sert également d'étalon de performance pour les autres techniques.
%
La seconde et la troisième technique sont basées sur la traduction dynamique,
qui utilise une représentation intermédiaire pour le résultat du décodage.
Les représentations intermédiaires des instructions décodées sont
stockées dans un cache et réutilisées quand les mêmes instructions
doivent être ré-exécutées.
%
La dernière méthode \emph{traduction dynamique avec spécialisation}
combine traduction dynamique et \emph{évaluation partielle}.
Cette dernière est une technique bien connue en compilation optimisante.
L'idée est de traduire un programme $P$ appliqué à une donnée spécifique $d$
en un programme spécifique plus rapide $Pd$.
On peut utiliser l'évaluation partielle en simulation pour spécialiser 
une instruction en une instruction plus simple,
à partir des données connues au moment du décodage.
Le décodeur de \simsoc implémente l'évaluation partielle.
Au moment du décodage, la traduction dynamique établit une correspondance
entre les instructions binaires et leurs spécialisations partiellement évaluées.
Bien que la spécialisation des instructions induise un surcroît
de consommation mémoire, ce dernier reste raisonnable rapporté
aux tailles mémoire disponibles sur les serveurs actuels.

Les technologies utilisées dans le simulateur de jeu d'instructions 
de \simsoc sont détaillées dans~\cite{ossc09}.



\subsubsection*{Performances}

Le module ARM de \simsoc réalisant le jeu d'instruction de l'architecture
ARMv5 a été écrit à la main.
Le simulateur peut simuler le circuit commercial SPEAr Plus600,
un SoC réalisé par ST Microelectronics comportant un système à double 
c{\oe}ur et plus de 40 composants supplémentaires,
ainsi que le circuit AM1705 de Texas Instruments.
Le simulateur peut émuler le contrôleur d'interruptions,
le contrôleur de mémoire,
le contrôleur de mémoire série,
le contrôleur Ethernet,
et tous les périphériques nécessaires pour amorcer Linux.
Exécuter le noyau Linux sur le simulateur de SPEAr Plus
est un moyen de tester et mettre au point le simulateur.
Tout d'abord le binaire du noyau Linux compressé est lu
sur la mémoire série, décompressé, puis Linux est démarré.
La simulation de ce processus ne prend que quelques secondes.
Le contrôleur Ethernet peut connecter à travers le protocole TCP/IP
plusieurs simulateurs du même SoC s'exécutant ou non sur la même machine.
Ce simulateur mature pour l'architecture ARMv5 était terminé
avant le début de notre projet de certification,
et deux autres simulateurs de jeu d'instruction pour PowerPC
et MIPS étaient également développés.


\subsubsection*{De  l'ARMv5 à l'ARMv6}

Pour cette thèse nous avons décidé de considérer la version suivante
(ARMv6) de l'architecture ARM, qui représente une montée en performances
depuis les c{\oe}urs ARMv5.
Pour l'essentiel, ARMv6 est compatible en arrière avec ARMv5.
Voici les nouveautés apparues dans l'architecture ARMv6.

\begin{itemize}
\item
  Le jeu d'instructions a été étendu par de nouvelles instructions
  dans six domaines: des instructions média, des instructions de multiplication,
  des instructions de contrôle et DSP,
  des instructions de chargement et stockage en mémoire,
  des instructions indéfinies et
  des instructions inconditionnelles.
  Heureusement, toutes les instructions ARMv5 obligatoires sont également
  instructions ARMv6 obligatoires.
  Pour les utilisateurs du simulateur, un code applicatif compilé pour
  l'ARMv5 s'exécute également sur l'ARMv6.
  Si un utilisateur souhaite bénéficier des nouvelles instructions V6,
  il doit recompiler son programme dans le nouvel environnement.
\item 
  Le mode \emph{Thumb} (instructions sur 16 bits) a changé.
  Les instructions  \emph{Thumb} de l'ARMv5 ne sont pas portables vers
  \emph{Thumb2} (ARMv6+), 
  la compatibilité arrière n'est pas complètement assurée.
\end{itemize}


\subsection*{Contributions}
\addcontentsline{toc}{subsection}{Contributions}

Dans ce travail nous avons développé une preuve de correction 
d'une partie du simulateur \simsoc.
Au delà de la certification d'un simulateur,
il s'agit d'une nouvelle expérience en certification de programmes
non triviaux écrits en C.
Contrairement aux approches répandues,
nous n'utilisons pas une sémantique axiomatique
mais une sémantique opérationnelle,
en l'occurrence celle définie dans le projet \compcert.

Nous définissons une représentation du jeu d'instruction ARM
et de ses modes d'adressage formalisée en Coq,
grâce à un générateur automatique prenant en entrée
le pseudo-code des instructions issu du manuel de référence ARM.
Nous générons également l'arbre syntaxique abstrait \compcert
du code C simulant les mêmes instructions au sein de \simlight,
une version allégée de \simsoc.
Une version textuelle de \simlight, avait été auparavant développée
comme un composant de \simsoc par C.~Helmstetter~\cite{rapido11}.

À partir de ces deux représentations Coq,
nous pouvons énoncer
et démontrer la correction de \simlight,
en nous appuyant sur la sémantique opérationnelle définie dans \compcert.
Nos premiers résultats dans cette direction sont décrits dans~\cite{2011first}.
Cette méthodologie a ensuite été appliquée à au moins une instruction
de chaque catégorie du jeu d'instruction de l'ARM.

Au passage, nous avons amélioré la technologie
disponible en Coq pour effectuer des \emph{inversions},
une forme de raisonnement utilisée intensivement
dans ce type de situation~\cite{2013itp}.

Nos contributions supplémentaires comprennent un générateur de
décodeur pour les instructions ARM, également basé sur l'analyse du
manuel de référence de l'ARM,
et un générateur de tests pour le décodeur d'instructions,
qui peut générer des tests massifs couvrant toutes les instructions ARM.

\newpage
\section*{Conclusion}
\addcontentsline{toc}{section}{Conclusion}

Nous avons développé la certification d'une partie de \simlight,
un simulateur du jeu d'instructions ARM,
en utilisant la sémantique opérationnelle du langage C fournie par le projet \compcert.
Les preuves de correction ont été effectuées à l'aide de l'assistant à la preuve Coq.
Une grande partie de la spécification Coq et du modèle du simulateur ont été produits
automatiquement à partir du pseudo-code disponible dans le manuel de référence ARM.
Une technique de preuve Coq pour effectuer des \emph{inversions} a été introduite
pour résoudre des étapes de preuve administratives de manière plus satisfaisante
que par les tactiques standard de Coq comme \inversion.
Les termes de preuve engendrés par notre \hcinv ont par ailleurs une taille
bien plus petite qu'avec \inversion,
ce qui améliore la compilation Coq et fluidifie l'interaction avec Coq lors de la mise
au point des scripts.
Par ailleurs, nous avons construit un générateur de tests pour le décodeur 
d'instructions ARM, qui génère des tests massifs couvrant toutes les instructions ARM.

Dans la suite nous tirons quelques enseignements de notre usage 
de la sémantique opérationnelle pour démontrer la correction de \simlight,
quant à la faisabilité de cette approche pour la preuve de programmes C en général.
Nous donnons quelques éléments chiffrés sur le développement de \simsoccert
et précisons la base de confiance de ce projet.
Nous terminons par quelques prospectives.


\subsection*{L'approche sémantique opérationnelle en preuve de programmes C}

La technique de certification employée pour \simlight est basée sur la sémantique
opérationnelle de C fournie par \compcert. La représentation formelle des
programmes C pour chaque instruction ARM peut être obtenue à partir 
de l'AST représentant le pseudo-code de deux façons :
soit en le traduisant en un AST \compcert C, 
soit en le traduisant en une forme textuelle C,
qui est ensuite donnée en entrée à l'analyseur syntaxique de \compcert C.
Dans nos expériences, les deux approches donnaient des résultats équivalents.
\compcert C supporte un sous-ensemble de C qui est suffisamment riche
pour programmer la simulation des instructions de l'ARM dans \simlight.

Les preuves de correction ont été effectuées dans l'assistant à la preuve Coq.
Dans cette approche à la certification de programmes C, les étapes de preuve Coq
ne sont pas simples. 
Cependant nous avons pu considérer des programmes C ayant une spécification
complexe et de grande taille, utilisant le pouvoir expressif de Coq.
À partir de notre travail, nous pouvons conclure que l'approche
par sémantique opérationnelle est utilisable.

Les étapes de preuve liées à la sémantique de \compcert C peuvent être
considérablement simplifiées par la définition de tactiques dans le langage Ltac de Coq.
Notre premier script de preuve pour l'instruction ADC était long de 
plusieurs milliers de lignes.
Nous avons lors identifié les séquences répétées et commencé à définir
nos propres tactiques en Ltac,
ce qui a permis de raccourcir considérablement les scripts de preuve. 
La seconde version du script pour la preuve de correction de ADC
était environ trois fois plus petite que la première.
Dans la conception de ces tactiques, nous n'avons pas recherché la généralité.
Cependant, comme beaucoup d'instructions ARM d'une même catégorie
se ressemblent, nos tactiques peuvent en fait être réutilisées.

Dans la section~\ref{sec:tactic}, nous avons également introduit des tactiques
générales pour \simsoccert,
notamment pour trouver des fonctions dans le modèle mémoire de C, 
réutiliser des opérations de \emph{load/store}, etc.
Ces tactiques ne sont pas spécifiques à \simlight,
elles sont seulement relatives à la sémantique que \compcert C et
aux opérations en mémoire. 
Il en va de même pour notre technique d'inversion:
elle a été instanciée pour les besoins de \simsoccert comme une tactique
\hcinv dédiée aux relations inductives définies dans \compcert
(voir la section~\ref{ssec:invssc}).
Cependant ces tactiques peuvent être réutilisées dans d'autres projets
suivant la même approche pour démontrer la correction de programmes \compcert C,
par exemple le projet CCCBIP qui a démarré récemment dans note groupe et vise
à construire une compilateur certifiant prenant en entrée un langage
de haut niveau à base de composants pour les systèmes embarqués (BIP),
avec \compcert C comme cible intermédiaire.



\subsection*{Inversion sur mesure}

Notre inversion ``sur mesure'' (\emph{hand crafted})
présentée au chapitre~\ref{cpt:inv} a été expérimentée sur des preuves
de grande taille reposant sur de grosses relations inductives définies
indépendamment dans le projet \compcert.
Cela a joué un rôle crucial pour le succès de cette approche 
aux preuves correction de programmes C,
et le gain en flexibilité obtenu par \hcinv a été exploité pour
produire des preuves bien plus petites, robustes et faciles à gérer.

En l'état, ce n'est pas une tactique complètement automatique
comme la tactique originale \inversion.
Nous pensons qu'une telle automatisation pourrait être réalisée
en interagissant avec les mécanismes internes de Coq.
Cela pourrait être effectué pour des raisons d'efficacité et
serait appréciable notamment dans les cas où les sous-buts engendrés
pourraient être résolus automatiquement,
ou sans qu'il soit besoin de référencer les noms produits par l'inversion.

Mais dans un projet comme \simsoccert,
qui comporte une spécification de grande taille
et où les preuves nécessitent des mises au point fines,
les interactions entre l'utilisateur et l'assistant à la preuve
ne peuvent pas être évitées.
En général, dans de telles situations, les énoncés mettent en {\oe}uvre
des définitions arbitrairement complexes,
et on ne peut faire l'hypothèse que des procédures de décision
pourront tout résoudre.
Le problème est alors de fournir des mécanismes appropriés,
de sorte à faciliter l'écriture des scripts de preuve et l'interaction
avec l'assistant de preuve.
Nous pensons que notre technique d'inversion sur mesure est un bon outil
de ce point de vue :
elle est suffisamment souple pour l'utilisateur,
les situations pratiques peuvent être gérées
en contrôlant totalement les scripts
et des améliorations intéressantes des scripts
sont plus faciles à concevoir.

Mentionnons une autre application possible de cette technique.
Des inversions sont parfois nécessaires dans l'écriture d'une fonction
dont les propriétés seront établies ultérieurement
(à l'opposé du style où l'on fournit une spécification à la Hoare
monolithique et exhaustive et un générateur de VC comme Program).
Dans ce contexte, une utilisation simple du moteur de preuve
et de la tactique \inversion a tendance à générer des termes
extrêmement gros et compliqués à gérer.
Notre technique devrait s'avérer très utile dans ce genre de situation.

\subsection*{Taille du développement}

\begin{table}[ht]
  \centering
  \begin{tabular}{|l|r@{~}|}
    \hline
    Manuel de référence ARM original (txt)  & 49655 \\
    Analyseur ARM vers AST OCaml         & 1068 \\
    Générateur (Simgen) pour l'ARM         &   10675 \\
    Générateur de spécifications pour SH4      & 737 \\
    Bibliothèques générales C sur l'ARM         & 1852 \\
    Bibliothèques générales Coq sur l'ARM         & 1569 \\
    Code C généré pour les opérations ARM \simlight   & 6681 \\
    Code Coq généré pour les opérations ARM   & 2068 \\
    Code Coq généré pour le décodeur ARM  & 592 \\
    Projection   & 857 \\
    Script de preuve pour ADC (2011)    & 3171 \\
    Script de preuve pour ADC (2012)    & 1204 \\
    Définition de \hcinv       &551\\
    Autres tactiques       &185\\
%% Xiaomu: please complete
    Script de preuve pour fonctions auxiliaires   & 856 \\
    Script de preuve pour BL (2012)   & 437 \\
    Script de preuve pour LDRB (2012)   & 170 \\
    Script de preuve pour MRS (2012)   & 322 \\
    \hline
  \end{tabular}
  \smallskip
  \caption{Tailles (en nombre de lignes)}
  \label{tab:sizesfr}
\end{table}

La table~\ref{tab:sizesfr} indique la taille de notre développement.
La taille du générateur atteint presque le nombre total de lignes
des parties générées pour l'ARMv6.
Mais il faut noter qu'il s'agit ici de la version courante refaite
par F.~Tuong pour généraliser le procédé à d'autres processeurs.
Actuellement, en dehors de l'ARM, cette chaîne de génération est
également appliquée au manuel de référence du processeur SH4 où,
à la place d'un pseudo-code spécifique,
les instruction sont décrites dans la syntaxe du langage C.

On peut également noter que le code généré pour le simulateur de
jeu d'instructions occupe 50\:\% du modèle formel Coq,
et presque 70\:\% du simulateur C.

Bien que le gain en volume puisse être considéré comme relativement faible,
nous pensons que cette approche est néanmoins valable
étant donnée la nature répétitive des instructions.

En ce qui concerne l'effort de preuve,
la première expérience a porté sur la correction de l'instruction ADC
et cela nous a pris un mois.
Le nombre de lignes Coq pour le script de preuve était alors
assez grand, environ 3200 pour cette première version
(surtout si l'on compare aux 11 lignes du pseudo-code correspondant
dans le manuel de référence).
À ce stade nous n'avions pas encore développé de tactiques utilisateur.
Ensuite, en utilisant \hcinv et nos autres tactiques spécifiques,
en dehors des gains en taille et en maintenabilité,
le temps de développement pour la preuve d'une instruction est bien plus
faible,
moins d'une semaine pour une instruction du même degré de complexité que ADC.
A présent, nous avons une preuve de correction pour 11 instructions,
une dans chaque catégorie d'instructions pour l'ARM.
Elles sont présentées dans la table~\ref{tab:prvinstfr}.

\begin{table}[ht]
  \centering
  \begin{tabular}{|l|l|}
    \hline
    Catégorie & Nom de l'instruction \\
    \hline
    branchement & BL \\
    calcul & ADC \\
    multiplication & MUL \\
    addition et soustraction en arithmétique parallèle  & QADD16 \\
    instruction étendue & UXTAB16 \\
    arithmétique divers & CLZ \\
    accès au registre de status  & MRS \\
    chargement et stockage & LDR \\
    chargement et stockage multiple & LDM \\
    sémaphore & SWP \\
    \hline
  \end{tabular}
  \smallskip
  \caption{Instructions ARM avec une preuve de correction}
  \label{tab:prvinstfr}
\end{table}

\subsection*{Base de confiance du code}

Nos preuves dépendent de plusieurs outils développés ailleurs:
l'assistant à la preuve Coq,
le compilateur OCaml et le compilateur certifié \compcert C.
La base de confiance de ces outils doit être considérée indépendamment.
En ce qui concerne Coq, il s'agit essentiellement du noyau.

Ensuite, la base de confiance comprend la version formelle
du manuel de référence servant à formuler les théorèmes
de correction:
il s'agit de définitions Coq produites manuellement et automatiquement,
selon le procédé décrit en figure~\ref{fig:arch}.
Une alternative pourrait être de remplacer les définitions Coq produites
automatiquement par le manuel de référence textuel (corrigé par nos soins)
et la chaîne de génération de code Coq.

La base de confiance comprend enfin
les projections définies en Coq entre la représentation du code
de \simlight sous forme d'AST \compcert C
et notre modèle Coq abstrait.

\subsection*{Travaux futurs}

La prochaine étape serait d'étendre le travail effectué sur
\texttt{ADC} et les autres opérations données dans la table~\ref{tab:prvinstfr}
au jeu d'instruction entier.
Nous somme confiants dans le fait que le travail correspondant peut être effectué
beaucoup plus vite.
En particulier, des lemmes portant sur 14 fonctions de la bibliothèque
sont déjà disponibles.
Il reste 71 fonctions semblables dans la librairie.

Les AST des fonctions de la bibliothèque sont obtenus au moyen de l'analyseur
syntaxique pour \compcert C.
Ces AST sont actuellement regroupés à la main avec les AST des instructions,
pour résoudre le problème technique mentionné page~\pageref{page:libfunast}.
Il serait préférable d'introduire un mécanisme permettant
de trouver automatiquement les fonctions appelées dans les AST produits
par l'analyse syntaxique et de générer des numéros de blocs inutilisés
pour les identificateurs correspondants.

Nous avons également tenté d'écrire une version Coq (fonctionnelle)
du décodeur,
mais des améliorations importantes sont nécessaires pour le rendre utilisable.
La version courante est basée sur un filtrage géant,
qui considère les 32 bits d'une instruction binaire
dans un ordre soigneusement mis au point.
Nous avons commencé à concevoir une meilleure version de ce décodeur,
en considérant la sémantique des champs de bits.
%
Cela fait, des preuves sur le décodeur pourront également être considérées.
Les outils d'extraction automatique du codage des instructions décrit dans
le manuel de référence sont déjà disponibles.
%
Enfin, la correction de la boucle de simulation
(essentiellement : répéter le décodage et l'exécution des opérations)
pourra être prouvée.

Dans une autre direction, notre méthodologie peut être réutilisée sur
d'autres processeurs, comme le SH4.

Par la suite, on pourra également considérer \simlight 2,
qui comporte plusieurs optimisations en vue d'accélérer
la simulation.
La différence la plus importante est l'application d'une
méthode de \emph{flattening} (voir \S~\ref{sec:codegen})
consistant à fusionner le code des modes d'adressage dans 
certaines instructions.
Le décodeur de \simlight 2 décode alors simultanément 
une instruction et son mode d'adressage.
Cela rend la définition en C plus simple que dans \simlight
et produit moins d'appels de fonctions.
Les preuves correspondantes pour le décodeur devraient donc êtr
plus simples pour \simlight 2 que pour \simlight.
%
Le corps des instructions est essentiellement le même dans \simlight 2
que dans \simlight.
La principale optimisation effectuée dans \simlight 2 consiste à
spécialiser certains paramètres aux valeurs effectives utilisées.
Ainsi, l'opération d'une instruction ARM est implémentée
par plusieurs fonctions dans \simlight 2, là où il n'y a qu'une seule fonction
dans \simlight ;
mais le code de ces fonctions est essentiellement le même,
ce qui donne bon espoir à la possibilité de réutiliser
les lemmes de correction existants pour \simlight,
en les reformulant et généralisant de manière adéquate,
de sorte que par instanciation on retrouve les lemmes
de correction attendus pour les fonctions correspondantes dans \simlight 2.

Notre équipe a récemment démarré un autre projet visant
l'implantation de logiciel certifié écrit en BIP,
un langage de haut niveau à base de composants dédié aux systèmes embarqués,
avec \compcert C comme langage intermédiaire.
Nous avons bon espoir que le travail présenté dans cette thèse
pourra être réutilisé.
%
Plus généralement,
notre réalisation de \hcinv pour \compcert peut être réutilisée
dans toute application visant à prouver la correction de programmes C
à partir de la sémantique opérationnelle définie dans \compcert.
Cependant, cette tactique doit évoluer en fonction 
des nouvelles versions de \compcert.



%%%%%%%%%%%%%%%%%%%%%%%%%%%%%%%%%%%%%%%%%%%%%%%%%%%%%%%%%%%%%%%%%%%%%%%%%
\selectlanguage{english}

%%% Local Variables: 
%%% mode: latex
%%% TeX-master: "thesis"
%%% End: 
