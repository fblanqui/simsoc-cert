\chapter{Simulation of ARMv6 in C}
\label{cpt:carm}

% \jf{Short introduction in 2 or 3 sentences}

% \jf{In particular comment section 2: optimized Simlight
% could have been considered but it is not the case.
% The conclusion (last chapter, Future work) should then say a word on
% extending this work to simlight 2.}

Here we introduce \simlight, our certification target.
\simlight is a light version of the simulator \simsoc,
which includes only the ARMv6 instruction set simulator
with a simplified memory model.

We also give a brief description of \simlight 2,
which includes several optimizations from \simsoc
to obtain a higher simulation speed.
The extension to the optimized version 2 will be discussed as future work in
conclusion Chapter~\ref{cpt:concl}.

% We have two versions of \simlight:
% version 1 is the current certification target,
% version 2 is based on version 1 but several optimizations were applied
% to obtain a higher simulation speed.

% Until now we only considered \simlight version 1,
% which is the closest to the formal description.
% The extension to the optimized version 2 will be discussed as future work in
% conclusion Chapter~\ref{cpt:concl}.

\selectlanguage{french}
\section*{Résumé}

\begin{resume}
Ce chapitre est consacré à \simlight, la cible que nous cherchons à certifier.
\simlight est une version allégée du simulateur \simsoc,
qui ne contient que le simulateur de jeu d'instructions,
avec un moèle mémoire simplifié.

Nous donnons aussi une brève description de \simlight 2,
qui intègre plusieurs optimisations utilisées dans \simsoc
afin d'accélérer la simulation.
L'extension de notre travail de certification à la version 2 optimisée
est considérée en perspective dans la conclusion,
au chapitre~\ref{cpt:concl}.
\end{resume}

\selectlanguage{english}

\section{Simlight}
\label{sec:slv6}

Similarly to the Coq formal model,
% JF -> XM: "a data structure in C specification" does not work
% because C is a dirty prgramming language with dirty semantics
% A best ACSL is a specification language related to C
\simlight contains a data structure in C to represent the ARMv6 processor.
As the structure of the processor did not change between ARMv5 and ARMv6,
this data structure was copied from the previous version of \simsoc for ARMv5.
%
It was designed to optimize execution,
which makes it rather different from the formalization in Coq that keeps things
as simple as possible, and as close as possible to the reference manual.
% JF : I added then commented out the following, because it is said below.
% In particular, it contains redundant or additional fields, as well as aliases.

% XM
% In Figure~\ref{fig:procc}, the C definition of the processor state is showed.
% JF
The C definition of the processor state is given in Figure~\ref{fig:procc}.

\begin{figure}
\begin{alltt}
struct SLv6_Processor \{
  struct SLv6_MMU *mmu_ptr;
  struct SLv6_StatusRegister cpsr;
  struct SLv6_StatusRegister spsrs[5];
  struct SLv6_SystemCoproc cp15;
  size_t id;
  uint32_t user_regs[16];
  uint32_t fiq_regs[7];
  uint32_t irq_regs[2];
  uint32_t svc_regs[2];
  uint32_t abt_regs[2];
  uint32_t und_regs[2];
  uint32_t *pc; /* = &user_regs[15] */
  bool jump;
\};
\end{alltt}
\caption{ARM Processor state in C}
\label{fig:procc}
\end{figure}

Similarly,
the data structure \texttt{SLv6\_Processor} contains
the most important components of the ARM processor:
a pointer to the location of the structure representing the Memory Management Unit (MMU),
a status register structure for CPSR,
an array for the status register structure of SPSR,
a structure for CP15 (SCC),
a field for the processor id (useful when there is more than one core),
six arrays for registers of each processor mode,
one field for PC,
and a boolean field \texttt{jump}
which indicates whether the instruction modifies the PC or not.

For a better illustration, the C definition of the status register structure is
given in Figure~\ref{fig:stregc}.

\begin{figure}
\begin{alltt}
struct SLv6_StatusRegister \{
  bool N_flag; /* bit 31 */
  bool Z_flag;
  bool C_flag;
  bool V_flag; /* bit 28 */
  bool Q_flag; /* bit 27 */
  bool J_flag; /* bit 24 */
  bool GE0; /* bit 16 */
  bool GE1;
  bool GE2;
  bool GE3; /* bit 19 */
  bool E_flag; /* bit 9 */
  bool A_flag;
  bool I_flag;
  bool F_flag;
  bool T_flag; /* bit 5 */
  SLv6_Mode mode;
  uint32_t background; /* reserved bits */
\};
\end{alltt}
\caption{ARM status register structure in C}
\label{fig:stregc}
\end{figure}

In order to gain high speed for the simulator,
the processor type has been designed to have
several redundant fields; for example, the PC field is a pointer
to the 15th register in user register array.
Indeed, the PC field is significant as it allows
to judge whether the execution is branched or not,
so that the running program can be split into code blocks.
And the PC field is referred to many times during execution.
% XM
% This part is related to a very important optimization method
% JF
There is another important optimization method
in \simsoc simulation ~\ref{sec:simsoc}.
In order to optimize execution time, the data structures used to
represent the processor state do not reflect the hardware structure.
For example, whereas a bit is used in hardware to store a flag,
a boolean variable is used to represent that bit in the C code.
For example the 4 bits in the status register that indicate the traditional
N, Z, C, V flags (Negative, Zero, Carry and oVerflow) for condition code,
are represented by 4 booleans variables.
Similarly, an array of status registers is used to represent the
status in the different modes, indexed by the current mode.
% The processor mode is not directly placed inside the data structure
% for the processor state, but for the status register.
% And there is another way to represent what is the current processor mode
% by looking at which register array is currently used.
% For each processor mode there is a corresponding register array.
% This is because the simulator never uses mode as a condition,
% it can go straight to the field that want to look for to save time.
% By contrast, the status register is not a simple 32-bit integer.
% It is also a data structure containing every significant bit of the register.
It means that the pseudo-instructions code that manipulate these data
structures must be translated by appropriate C code that accesses the
corresponding data,

% These flags in the Current Program Status Register (CPSR) are tested by most of the
% instructions to determine whether the instruction is going to be executed or skipped.
% All the other bits are seen as reserved or \texttt{background}.
% Since they are reserved for future extension and must be read as zero,
% any write operation on them should be ignored.
% For future compatibility, they must be written with the same value from the same bits.

As a result, the C code of the semantics functions has a structure
close to the pseudo-code in the reference manual but with additional
access functions to access the data in an optimized implementation.
Below is the example of the ADC instruction:

\begin{alltt}
\small
/* ADC */
void ADC(struct SLv6_Processor *proc,
    const bool S,
    const SLv6_Condition cond,
    const uint8_t d,
    const uint8_t n,
    const uint32_t shifter_operand)
\{
  const uint32_t old_Rn = reg(proc,n);
  const uint32_t old_CPSR = proc->cpsr;
  if (ConditionPassed(&proc->cpsr, cond)) \{
    set_reg_or_pc(proc,d,((old_Rn + shifter_operand) + old_CPSR.C_flag));
    if (((S == 1) && (d == 15))) \{
      if (CurrentModeHasSPSR(proc))
        copy_StatusRegister(&proc->cpsr, spsr(proc));
      else
        unpredictable();
    \} else \{
      if ((S == 1)) \{
        proc->cpsr.N_flag = get_bit(reg(proc,d),31);
        proc->cpsr.Z_flag = ((reg(proc,d) == 0)? 1: 0);
        proc->cpsr.C_flag = CarryFrom_add3
                              (old_Rn,shifter_operand,old_CPSR.C_flag);
        proc->cpsr.V_flag = OverflowFrom_add3
                              (old_Rn,shifter_operand,old_CPSR.C_flag);
      \}
    \}
  \}
\}
\end{alltt}
% VJ replacement


As the C language accepted by Compcert is a subset of the full ISO C language,
the generator has been constructed such that it only generates C code
in the subset accepted by the compcert compiler.

Nonetheless it can be compiled with other C compilers such as GCC
to obtain better performance. Even though in this case, the resulting
machine code is not guaranteed to be correct (there are well known
GCC optimization bugs...), at least the original C code has been
proven by our technique to be conformant with the ARM semantics.

The ARM V6 code generator not only generates the semantics functions,
it also generates the decoder of binary instructions supported in V6
architectures. This decoder is obtained by compiling the opcodes
information. The generated decoder is probably not optimal in
performance, but as \simsoc uses a cache to store the decoded
instructions, the performance penalty is marginal.
% The decoder in \simlight also uses the encoding table of each instruction
% as reference. The principle is to use an \texttt{if...} statement to filter
% the impossible cases.

% To be precise, \simlight (Version 1) is in \compcert-C, the
% large subset of the C language considered in the \compcert project.
% Then the whole simulator can be compiled with the certified compiler \compcert.
% In this way, errors coming from an ordinary C compiler can be avoided
% and the compilation result is also reliable.
% Generating directly \compcert-C code makes sure that none
% unaccepted code has been ignored by the compiler.

\section{Optimization on Simlight version 2}
\label{sec:slv62}

In this section, we introduce the second version of \simlight,
that include optimizations to reduce simulation time.
The optimization methods can be categorized as follows:
\begin{itemize}
\item
  \textit{flattening} is used in \simlight version 2,
  in order to merge some instructions with their addressing mode.
  The result of \textit{flattening} can improve the simulation speed.
  How to perform instructions flattening is introduced in Section~\ref{sec:codegen}.
\item
% modified vy VJ
  Partitioning the semantics function into hot and cold ones. C
  compilers now supports the {\em hot} and {\em cold} attributes on
  functions. When a function is declared hot, the compiler generates
  code that minimizes execution time. When it is cold, it minimizes
  code size. It also generates directives for the linker to group the
  hot and cold functions together to increase program locality. The
  temperature information (i.e. hot or cold) is obtained by running
  the program on sample input to generate profiling data, such as
  obtained with the GPROF profiling tool.  Based on profiling data,
  the \simsoc generator can partition the functions between cold and
  hot in order to benefit from these compiler optimizations.
  % Partitioning the semantics function into hot and cold. That means the
  % compiler does not inline the functions which are not called frequently
  % and gather them to be ``cold''. In contrary, the hot call sites are
  % inlined in order to keep the code size small and shorten the
  % compilation time or other benefits from inlining functions.
  % Keeping frequently used functions together in ``hot'' group, the
  % instruction cache has been less polluted than cold functions.
  % GCC with a certain option can direct the partition of functions.
  % A parameter ``wgt'' (weight) is computed by the gcc compiler,
  % which is a space-separated list of non-negative integers of
  % length the number of instructions. It is given by \simsoc experiment.
  % And it is required by GCC as the basis of hot/cold semantics function partition.
  % If ``wgt'' is not up to date, then the hot/cold partition is weak.
% \item
%   Another way of grouping instruction, by the list of parameters.
%   The ones with the same parameter list will be grouped together.
% \item
%   Shorten the head file of instruction set specifications because they are
%   identical to each other.
% VJ remark, these changes improve readability of the code
% and bug fixing but not performance
\item
  Specialize the instruction of boolean parameter values.
\item
  Remove the instructions about coprocessor because the coprocessor is not
  supported yet. It can save time in encoding and decoding phases.
\end{itemize}



%%% Local Variables:
%%% mode: latex
%%% TeX-master: "thesis"
%%% End:
