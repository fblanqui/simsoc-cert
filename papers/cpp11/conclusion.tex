\section{Conclusion}
\label{sec:conclusion}

The trust we may have in our result depends on
the faithfulness of its statement with relation to 
the expected behavior of the simulation of \texttt{ADC} in \simlight.
It is mainly based on 
the manually written Coq and C library functions, 
the translators written in OCaml described in 
Section~\ref{sec:overallarchi}
(including the pretty-printer for Coq),
the final phase of the Compcert compiler,
and the formal definition of $\mathit{proc\_state\_related}$.



The current development is available online\footnote{%
\label{f:site}
\url{http://formes.asia/media/simsoc-cert/}}.
Figures on the size of our current development are given in
Table~\ref{tab:sizes}.

\begin{table}[t]
  \centering
  \begin{tabular}{|l|r@{~}|}
    \hline 
  % "wc -l" tested with svn revision : 1602
    Original ARM ref man (txt)                                             & 49655 \\ % wc -l arm6/arm6.txt
    ARM Parsing to an OCaml AST                                            & 1068 \\ % wc -l arm6/parsing/*{ml,sh}
    Generator (Simgen) for ARM and SH with OCaml and Coq pretty-printers   & 10675 \\ % wc -l simgen/*{ml,v,mly,mll}
    Generated C code for Simlight ARM operations                           & 6681 \\ % wc -l arm6/simlight/slv6_iss.c
    General Coq libraries on ARM                                           & 1569 \\ % wc -l arm6/coq/*v
    Generated Coq code for ARM operations                                  & 2068 \\
    Generated Coq code for ARM decoding                                    & 592 \\
    Proof script on \texttt{ADC}                                           & 1461 \\ % wc -l devel/xiaomu/test_csem_draft.v
    \hline 
  \end{tabular}
  \smallskip
  \caption{Sizes (in number of lines)}
  \label{tab:sizes}
\end{table}

In the near future, we will extend the work done on \texttt{ADC} 
to all other operations. 
The first step will be to design relevant suitable tactics,
from our experience on \texttt{ADC}, 
in order to shorten a lot the current proof and make it easier
to handle and to generalize.
We are confident that the corresponding work on 
the remaining ARM operations will then be done much faster,
at least for arithmetical and Boolean operations.

Later on, we will consider similar proofs for the decoder
-- as for the body of operations, it is already automatically
extracted from the ARM reference manual.
Then a proven simulation loop (basically, repeat decoding
and running operations) will be within reach.

In another direction, we also reuse the methodology based on
automatic generation of simulation code and Coq specification
for other processors. 
The next one which is already considered is SH4. 
In fact, the same approach as the ARMv6 has been followed, and a
similar Coq representation can currently be generated from the SH4
manual. Moreover, as the SH pseudo-code is simpler than the ARM, we are
hence impatient to work on its equivalence proof.


\subsection*{Acknowledgement}
We are grateful to Vania Joloboff and Claude Helmstetter for
their many explanations on SimSoC.
We also wish to thank the anonymous reviewers for their
detailed comments and questions. 

%%% Local Variables: 
%%% mode: latex
%%% TeX-master: "cpp"
%%% End: 
